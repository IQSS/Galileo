\documentclass[12pt,letterpaper]{article}
% === graphic packages ===
\usepackage{epsf,graphicx,psfrag}
% === bibliography package ===
\usepackage{natbib}
% === margin and formatting ===
\usepackage{setspace}
\usepackage{fullpage}
\usepackage{caption}
% === math packages ===
\usepackage[reqno]{amsmath}
\usepackage{amsthm}
\usepackage{amssymb,enumerate}
\newtheorem{Com} {Comment}
\newtheorem{prop}{Proposition}
\newtheorem{thm}{Theorem}
% === dcolumn package ===
\usepackage{dcolumn}
\newcolumntype{.}{D{.}{.}{-1}}
\newcolumntype{d}[1]{D{.}{.}{#1}}
% === additional packages ===
\usepackage{url}
\newcommand{\makebrace}[1]{\left\{#1 \right \} }
\newcommand{\determinant}[1]{\left | #1 \right | }
\newcommand{\ind}{1\hspace{-2.8mm}{1}}
\begin{document}


\section*{{\tt affinity} : Compute Similarity Between Documents}
\label{affinity} {\tt affinity} provides a matrix of affinities
between documents.  To improve the topics detected in the data, {\tt
affinity} allows users to discard unimportant words, remove stop
words, and compute a set of weights to identify important words. The
output from {\tt affinity} is provided to {\tt galileo} to uncover
topics.
\subsection*{Syntax}
\begin{verbatim}
aff.mat<- affinity(undergrad, tfidf=T, type='cosine', stop=T,
extra=F, extra.dat=NULL)
\end{verbatim}
\subsection*{Inputs}
\begin{itemize}
\item{{\tt undergrad}} An object from {\tt undergrad}, counting the
occurrences of stems in each document.
\item{{\tt tf-idf=T}} if true, tf-idf weights are used to calculate
the affinity between documents.
\item{{\tt type='cosine'}} Selects the measure of affinity between
documents.  Currently, only the cosine is implemented, but future
methods include euclidean (normal distribution), correlation, and
absolute deviation.
\item{{\tt stop=T}} if true, removes stop words from each document's
representation when calculating the affinity between documents.
\item{{\tt extra=T}} if true, there is an additional set of words to
be removed from each document's representation before calculating
the affinity between documents
\item{{\tt extra.dat}} if {\tt extra=T} then the additional words
(in a data frame) must be supplied to {\tt extra.dat}.
\end{itemize}

\subsection*{Details}
{\tt affinity} creates a matrix that measures the similarities
between a set of documents, which is supplied to {\tt galileo} to
group documents with similar topics.  Using the {\tt stop}, {\tt
tf-idf}, {\tt extra.dat} options allows the user to supply
information to help increase the likelihood {\tt galileo} will
uncover meaningful topics from the data.


\subsection*{Examples}
\begin{verbatim}
##gt is an object from undergrad

library(Galileo)

data(SenPress)

data(CommWords)

output<- affinity(gt, stop=T, tfidf=T, extra=T,
extra.dat=comm.words)

\end{verbatim}

\subsection*{Output Values}
if there are $k$ documents to be analyzed, {\tt affinity} returns a
$k \times k$ matrix of similarities between documents.


\subsection*{Contributors}
CONTRIBUTORS HERE.


\clearpage


\section*{{\tt galileo} : Methods for Topic Detection in Texts}
\label{galileo} The {\tt galileo} command provides a variety methods
for topic-detection in texts.  {\tt galileo} takes as an input a
matrix of affinities between documents and returns a vector of
cluster assignments between documents, which groups documents
together based upon the topic of the text.  This classification can
be used by {\tt mutinf} to label the clusters.
\subsection*{Syntax}
\begin{verbatim}
clust <- galileo(affinity, method=c('Affinity', 'Spectral'), p=NULL
, lambda=0.6, maxits=500, convits=50, num=NULL)
\end{verbatim}
\subsection*{Inputs}
\begin{itemize}
\item{{\tt affinity}} output from {\tt affinity} a matrix of
affinities (similarities) between a set of documents.
\item{{\tt method= c('Affinity', 'Spectral')}} selects the method of
text-clustering for topic detection.  {\tt Affinity} results in
text-clustering by approximate mixture of von-Mises distribution,
with estimation by Dueck and Frey's (2007) affinity propagation.
{\tt Spectral} computes the clusters using the two step clustering
procedure developed in Ng, Jordan, and Weiss (2002).
\item{{\tt p=NULL}} If {\tt Affinity} is selected, {\tt p} sets the
prior that each document forms a cluster center and therefore is a
prior on the number of topics.  Specifically, $p = \log
\text{Pr}(\text{doc}_i = \text{center}) + 1$, or the log probability
an arbitrarily chosen document is a cluster center.
\item{{\tt lambda=0.6}} Dampening coefficient for method {\tt
affinity}, must be between $(0.5,1)$.
\item{{\tt maxits=500}} is the maximum number of iterations for
affinity propagation to carry out before ceasing
\item{{\tt convits=50}} is the number of iterations necessary to
establish convergence.
\item{{\tt num}} sets the number of topics if {\tt Spectral} method
is selected
\end{itemize}

\subsection*{Details} {\tt galileo} implements two different topic
detection methods. The first, {\tt Affinity} estimates an
approximate mixture of von Mises-Fisher distributions using Dueck
and Frey's (2007) Affinity Propagation algorithm.  DISCUSSION of THE
APPROXIMATE MIXTURE OF VMF DISTRIBUTIONS.  Affinity propagation is a
special case of 'loopy belief-propagation'--an approximate
maximization approach for NP-Hard problems. The second method {\tt
Spectral} implements the spectral clustering method of Ng, Jordan,
and Weiss (2002). Spectral clustering methods have recently been
developed in Computer Science and provide a quick approach to
document clustering that have been shown to be competitive with
mixture modeling approaches.

\subsection*{Examples}
\begin{verbatim}
##output is an object from the affinity function

##Example code for Affinity propagation

cluster<- galileo(log(output), p = median(log(output))- 10,
lambda=0.6, method='Affinity')

##Example code for Spectral Clustering

cluster<- galileo(output, num=30, method='Spectral')

\end{verbatim}

\subsection*{Output Values} {\tt galileo} returns a vector of cluster
labels for each document.


\subsection*{Contributors} CONTRIBUTORS HERE.


\clearpage

\section*{{\tt mutinf} : Compute Cluster Labels using Mutual Information}
\label{mutinf} {\tt mutinf} determines a set of cluster labels using
output from {\tt galileo}.  Specifically, {\tt mutinf} computes the
information each stem $w$ provides for correct classification of
documents in topics.  The output from {\tt mutinf} and {\tt galileo}
can be passed to {\tt clustTable} to produce a \LaTeX ready table.

\subsection*{Syntax}
\begin{verbatim}
muts<- mutinf(cluster, undergrad, stop=T, extra=F, extra.dat=NULL)
\end{verbatim}
\subsection*{Inputs}
\begin{itemize}
\item{{\tt cluster}} is an object from {\tt galileo} containing the cluster
assignments for a set of documents.
\item{{\tt undergrad}} is an object from undergrad containing the representation
of each document.
\item{{\tt stop=T}} if true, removes stop words from each document's
representation when calculating the affinity between documents.
\item{{\tt extra=F}} if true, there is an additional set of words to
be removed from each document's representation before calculating
the affinity between documents
\item{{\tt extra.dat}} if {\tt extra=T} then the additional words
(in a data frame) must be supplied to {\tt extra.dat}.
\end{itemize}

\subsection*{Details} {\tt mutinf} produces a set of labels for
clusters of documents, regardless of the method used to compute the
cluster assignments. The mutual information measures how well a stem
$w$ predicts whether a document is assigned to a topic $k$.  The
mutual information will reach its maximum if a stem is only found in
documents that assigned to topic $k$ and reaches its minimum if the
stem does not help discriminate between the documents assigned to a
category and the documents not assigned to the category.



\subsection*{Examples}
\begin{verbatim}
##gt is an object from undergrad

library(Galileo)

data(SenPress)

data(CommWords)

output<- affinity(gt, stop=T, tfidf=T, extra=T,
extra.dat=comm.words)

cluster<- galileo(log(output), p = median(log(output))- 10,
lambda=0.6, method='Affinity')

muts<- mutinf(cluster, gt, extra=T, extra.dat=comm.words)


\end{verbatim}

\subsection*{Output Values} If there are $k$ topics and $W$ stems,
produces a $k \times W$ matrix, describing the mutual information of
each stem, for each topic.

\subsection*{Contributors} CONTRIBUTORS HERE.

\clearpage

\section*{{\tt clustTable} : Table for Displaying Cluster Output}
\label{clustTable} {\tt clustTable} takes the output from {\tt
galileo} and {\tt mutinf} and produces a \LaTeX ready table.

\subsection*{Syntax}
\begin{verbatim}
clustTable(cluster, mut.inf, num=5)
\end{verbatim}
\subsection*{Inputs}
\begin{itemize}
\item{{\tt cluster}} is an object from {\tt galileo} containing the cluster
assignments for a set of documents.
\item{{\tt mut.inf}} is an object from {\tt mutinf} describing the mutual information for each
word and for each cluster.
\item{{\tt num}} describes the number of stems used to describe each
topic.
\end{itemize}

\subsection*{Details} {\tt clustTable} orders the clusters from
largest to smallest. Then, each cluster is labeled according to the
{\tt num} stems with the largest mutual information for a given
cluster.



\subsection*{Examples}
\begin{verbatim}
##gt is an object from undergrad

library(Galileo)

data(SenPress)

data(CommWords)

output<- affinity(gt, stop=T, tfidf=T, extra=T,
extra.dat=comm.words)

cluster<- galileo(log(output), p = median(log(output))- 10,
lambda=0.6, method='Affinity')

muts<- mutinf(cluster, gt, extra=T, extra.dat=comm.words)

clustTable(cluster, muts, num=5)

\end{verbatim}

\subsection*{Output Values} A \LaTeX ready table

\subsection*{Contributors} Justin Grimmer and Gary King implemented
{\tt clustTable}.

\end{document}
