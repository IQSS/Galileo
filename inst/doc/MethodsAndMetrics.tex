\documentclass[11pt,letterpaper]{article}
% === graphic packages ===
\usepackage{epsf,graphicx,psfrag}
% === bibliography package ===
\usepackage{natbib}
% === margin and formatting ===
\usepackage{setspace}
%\usepackage{palatino}[1]
%\usepackage[T1]{fontenc}
%\usepackage{palatino}
%\renewcommand{\familydefault}{cmss}
%\usepackage{cmbright}
%\usepackage[T1]{fontenc}
%\usepackage{arev}
%\usepackage[LGR,T1]{fontenc} %% LGR encoding is needed for loading the package gfsneohellenic
%\usepackage[default]{gfsneohellenic}
%\usepackage{pslatex}
\usepackage{fullpage}
\usepackage{color}
% === math packages ===
\usepackage[reqno]{amsmath}
%\usepackage[pdftex]{hyperref}
\usepackage{amsthm}
\usepackage{amssymb,enumerate}
\usepackage[all]{xy}
\usepackage{lscape}
 \numberwithin{equation}{section}
% === dcolumn package ===
\usepackage{dcolumn}
\newcolumntype{.}{D{.}{.}{-1}}
\newcolumntype{d}[1]{D{.}{.}{#1}}
% === additional packages ===
\usepackage{url}
\title{Similarity and Cluster Methods Included in Galileo}
\author{Justin Grimmer and Gary King}
\begin{document}
\maketitle
\bibliographystyle{apsr}

\tableofcontents
\bigskip

We suppose throughout that we observe $N$ documents $(i=1, \hdots,
N)$.  Suppose that document $\boldsymbol{y}_i$ is a $w \times 1$
count vector, where the $j^{\text{th}}$ element of the vector counts
the number of occurrences of the $j^{\text{th}}$ word.  Collect the
documents into the $N \times 1$ matrix $\boldsymbol{Y}$.

\section{Similarity and Dissimilarity Methods}
Crucial to the clustering objective is the definition of similarity
between documents.  Some methods allow for an arbitrary distance
metric to be used between the documents, while others (mainly
statistical models) operate on an implied distance metric.  We
outline some of the most important metrics here.  \\
\indent Abstractly, a dissimilarity metric is a mapping from
$\mathcal{D}: Y \times Y \rightarrow \Re_+$.  We will denote the
dissimilarity metric $\text{M}$ evaluated at $\boldsymbol{y}_i$ and
$\boldsymbol{y}_j$ as $\mathcal{D}_{\text{M}} (\boldsymbol{y}_i,
\boldsymbol{y}_j).$
\subsection{Unnormalized Metrics}


\subsubsection{Manhattan} The Manhattan metric between
$\boldsymbol{y}_i$ and $\boldsymbol{y}_j$ is given by
\begin{eqnarray}
\mathcal{D}_{\text{Manhattan}}(\boldsymbol{y}_i,\boldsymbol{y}_j ) &
= & \sum_{k=1}^{w} |y_{i,k} - y_{j,k}  |. \nonumber
\end{eqnarray}
\subsubsection{Euclidean}
Perhaps the most widely used of dissimilarity metrics, the Euclidean
distance metric  is given by
\begin{eqnarray}
\mathcal{D}_{\text{Euclidean}}(\boldsymbol{y}_i,\boldsymbol{y}_j ) &
= & \sqrt{\sum_{k=1}^{w} (y_{i,k} - y_{j,k} )^{2} }. \nonumber
\end{eqnarray}
\subsubsection{Minkowski}
The Minkowski metric generalizes the Manhattan and Euclidean metric.
The core idea behind both the manhattan and euclidean metric is the
that the deviation along each coordinate is raised to the
$p^{\text{th}}$ power and then the $p^{\text{th}}$ root is taken of
the sum.  In other words, we can specify the Minkowski($p$) metric
as,
\begin{eqnarray}
\mathcal{D}_{\text{Minkowski}, p }(\boldsymbol{y}_i,\boldsymbol{y}_j
) & = & \left( \sum_{k=1}^{w} \left[y_{i,k} - y_{j,k} \right]^{p}
\right)^{1/p} \nonumber
\end{eqnarray}
\subsubsection{Maximum}
The maximum metric (or \emph{sup-norm}) is an extension of the
Minkowski metric.  If we allow $p \rightarrow \infty$ then,
\begin{eqnarray}
\mathcal{D}_{\text{maximum}}(\boldsymbol{y}_i,\boldsymbol{y}_j ) & =
& \max \left\{|y_{j,1} - y_{k,1} |, \hdots, |y_{j,w} - y_{k,w} |
\right\} \nonumber
\end{eqnarray}
\subsection{Normalized Metrics}
A problem with the previous methods is that they are
\emph{unbounded} or do not normalize the length of documents.  We
now introduce a class of metrics that attempt to remove the
influence of document length in the determination of dissimilarity.
\subsubsection{Canberra}
Define the index set for two documents $i$ and $j$
$\mathcal{Z}_{i,j}  \equiv \{k \in W | y_{i,k} > 0 \text{ or }
y_{j,k} > 0 \}.$  Then, we can define the canberra metric as
\begin{eqnarray}
\mathcal{D}_{\text{canberra}}(\boldsymbol{y}_i,\boldsymbol{y}_j ) &
= & \sum_{k \in \mathcal{Z}} \frac{|y_{i,k} - y_{j,k} | }{|y_{i,k} +
y_{j,k}  | } \nonumber
\end{eqnarray}

\subsubsection{Cosine/(Not Centered Pearson)}
The cosine (or not centered pearson) is a \emph{similarity} metric
between two documents. Define the \emph{length} of document $i$ as
$\text{Len}_i  =  \sqrt{\boldsymbol{y}_i^{'} \boldsymbol{y}_i^{'}
}.$  Then the cosine of the angle between two documents $i$ and $j$
is
\begin{eqnarray}
\cos (\boldsymbol{y}_i, \boldsymbol{y}_j) & = &
\frac{\boldsymbol{y}_i ^{'} \boldsymbol{y}_j }{\text{Len}_i
\text{Len}_j} \nonumber
\end{eqnarray}
And because $\boldsymbol{y}_{i,k} \geq 0$ for all $i$ and $k$, we
know that $\cos (\boldsymbol{y}_i, \boldsymbol{y}_j) \geq 0$.
Therefore, we can define the distance metric as
\begin{eqnarray}
\mathcal{D}( \boldsymbol{y}_i , \boldsymbol{y}_j )_{\cos} & = & 1-
\cos (\boldsymbol{y}_i, \boldsymbol{y}_j) \nonumber
\end{eqnarray}
\subsubsection{Correlation/(Centered Pearson) }
Similar to the cosine metric is the correlation metric.  Define
$\text{Var}(\boldsymbol{y}_i)  = \frac{1}{N-1}\sum_{k=1}^{w}
(y_{i,k}^2 - \bar{y}_{i} )^2 $.  Then, it is well known that
\begin{eqnarray}
\text{cor}(\boldsymbol{y}_i, \boldsymbol{y}_j) & = &
\frac{\sum_{k=1}^{w} (y_{i,k} - \bar{y}_i )(y_{j,k} - \bar{y}_j)}{
\sqrt{\text{Var}(\boldsymbol{y}_i) \text{Var}(\boldsymbol{y}_j) }}
\nonumber
\end{eqnarray}

Then we can define the associated distance metric as,
\begin{eqnarray}
\mathcal{D}_{\text{correlation}}(\boldsymbol{y}_i, \boldsymbol{y}_j)
& = & 1 - \text{cor}(\boldsymbol{y}_i, \boldsymbol{y}_j)  \nonumber
\end{eqnarray}

\subsubsection{Binary}
Recall from above that $\mathcal{Z}_{i,j}  \equiv \{k \in W |
y_{i,k} > 0 \text{ or } y_{j,k} > 0 \}.$.  Define $\mathcal{F}_{i,j}
\subset \mathcal{Z}_{i,j}$ as $ \mathcal{F}_{i,j} \equiv \{ k \in
\mathcal{Z}_{i,j} | y_{i,k} > 0 \text{ and } y_{j,k} = 0 | y_{i,k}
=0  \text{ and } y_{j,k} > 0 \}.$  Then we can define the distance
metric as
\begin{eqnarray}
\mathcal{D}_{\text{Binary} }(\boldsymbol{y}_i, \boldsymbol{y}_j) & =
& \frac{|\mathcal{F}|}{|\mathcal{Z}|} \nonumber
\end{eqnarray}

In words, the Binary metric first determines the coordinates where
either document $i$ or document $j$ have a word.  Then, among those
words, we count the coordinates where \emph{only one document has a
positive entry} or those words that appear in only one document. The
ratio of these two sets defines a distance metric.
\subsection{Ranking Methods}
\subsubsection{Spearman's Ranked Test}
\textbf{Note, the implementation of this in AMAP must be wrong} \\
Consider two vectors $\boldsymbol{y}_i$ and $\boldsymbol{y}_j$.
First, convert each data vector into a vector of ranks.  That is,
the biggest value in each vector is assigned the largest rank (the
length of the vector $\boldsymbol{y}_i$), the second largest value
in the vector will be assigned the second largest rank, and so on.
Call this new vector $\tilde{\boldsymbol{y}}_i$ and
$\tilde{\boldsymbol{y}}_j$.  Then, we can define the
\textbf{Spearman Ranked Correlation} as,
\begin{eqnarray}
\textbf{Spearman Ranked Correlation}(\boldsymbol{y}_i,
\boldsymbol{y}_j) & = & \text{cor}(\tilde{\boldsymbol{y}}_i,
\tilde{\boldsymbol{y}}_j ), \nonumber
\end{eqnarray}
or the spearman ranked correlation is just the spearman-centered
correlation between the vector of ranks.  Then, we can define the
appropriate distance as
\begin{eqnarray}
\mathcal{D}_{\text{Spearman Ranked Correlation}}(\boldsymbol{y}_i,
\boldsymbol{y}_j)  & = &  1 - \textbf{Spearman Ranked
Correlation}(\boldsymbol{y}_i, \boldsymbol{y}_j) \nonumber
\end{eqnarray}
\subsubsection{Kendall}
Need to fill in this section.  the formula is
\begin{eqnarray}
\frac{c - d }{\frac{n(n-1)}{2}} \nonumber
\end{eqnarray}
where $c$ is \emph{concordant} and $d$ is \emph{discordant} pairs.
The problem is that I can't find a good definition of those
quantities.  Once again, we'll take 1 - Kendall tau.  Compute using
{\tt cor.test(data[1,], data[2,], method='kendall') } in {\tt R}.
\subsection{Other Measures of Dissimilarity}
\subsubsection{Random Forest Distance}
The random forest metric is a method based on the use of random
forest technique but for unsupervised learning.  In the unsupervised
case, we take the original data set $\boldsymbol{Y}$, and then
sample an artificial data set from the columns of $\boldsymbol{Y}$.
The original data set is labeled class $1$ and the synthetic data
set is class $2$.  Random forest learning is now applied to this
(now labeled) data set.  The distance metric is then created by
observing how many times two observations are ``voted" into the same
classification. So, if denote the iterations of the random forest by
$t$, $(t = 1, \hdots, T)$ and $v_{i,j}^{t}$ is an indicator for $i$
and $j$ voted in the same class we have,
\begin{eqnarray}
\mathcal{D}(\boldsymbol{y}_i, \boldsymbol{y}_j) & = &
 1 - \sum_{t=1}^{T}\frac{v_{i,j}^{t}}{T}. \nonumber
 \end{eqnarray}

\subsubsection{Encoding Metrics (Asymmetric)}
This is an asymmetric similarity metric that is suggested in Dueck
and Frey's (2007) article on affinity propagation, and is based upon
the information theoretic costs of encoding each document, using
another document's vocabulary.  \\
\indent Consider two documents $\boldsymbol{y}_k$ and
$\boldsymbol{y}_j$.  Suppose that there are $n_k$ words in
$\boldsymbol{y}_k$  and $n_j$ words in  $\boldsymbol{y}_j$.  If
there are $N$ documents (indexed by $i = 1, \hdots, N$) then we say
there is a total of $n = \sum_{i=1}^{N} n_i$ in all the documents.
We can then define the encoding metric (em) similarity between
$\boldsymbol{y}_k$ and $\boldsymbol{y}_j$ as,
\begin{eqnarray}
em(\boldsymbol{y}_k, \boldsymbol{y}_j) & = & - \log\left(n_j\right)
 \times \text{No. words in $\boldsymbol{y}_j$ and
$\boldsymbol{y}_k$} + - \log \left( n \right) \text{No. words in
$\boldsymbol{y}_j$ not in $\boldsymbol{y}_k$}. \nonumber
\end{eqnarray}
Of course $em(\boldsymbol{y}_k, \boldsymbol{y}_j)$ is not
necessarily equal to $em(\boldsymbol{y}_j, \boldsymbol{y}_k)$.  We
can alter the similarity metric to a distance metric with a simple
transformation,
\begin{eqnarray}
\mathcal{D}_{em}(\boldsymbol{y}_k, \boldsymbol{y}_j ) & = & \max_{i,
m}em(\boldsymbol{y}_i, \boldsymbol{y}_m)  - em(\boldsymbol{y}_k,
\boldsymbol{y}_j). \nonumber
\end{eqnarray}
\section{Clustering Methods}
Using the similarity and dissimilarity metrics from the previous
sections, we can define clustering methods.  The basic
(optimization) objective of clustering is to group together
documents with similar content and placing documents with different
content in different clusters.  Obviously, the crucial element of
the definition is ``different", which is why we use the similarity
and dissimilarity metrics.  \\
\indent But a second crucial element is identifying the optimal
groupings.  The clustering objective is an NP-Hard problem, with the
number of potential clusters growing quickly as the number of
observations increase.  Therefore, the clustering methods must
obtain an approximate method to quickly obtain a clustering of the
documents. \\
\indent Throughout, we will assume that our primary quantity of
interest is the cluster membership of each document.  If we use a
\emph{hard clustering} our quantity of interest is a set of cluster
labels $\boldsymbol{c} = (c_1, \hdots, c_N)$, where $c_i$ denotes
which of the $k$ clusters observation $i$ is assigned.
Alternatively, we might allow \emph{soft-clustering} which allows
documents to partially belong to each of the clusters.  In this case
$\boldsymbol{c}_i$ is a $K \times 1$ vector, where $c_{k,1}$
describes the posterior probability of document $i$ belonging to the
$k^{\text{th}}$ category.
\subsection{Exemplar Based Methods}
Exemplar based methods are hard-clustering techniques.  To aid in
dividing the data into clusters (equivalently obtaining the $N
\times 1$ vector $\boldsymbol{c}$), we suppose that each cluster is
characterized by a \emph{cluster-center} $\boldsymbol{\mu}_k$ which
is a $w \times 1$ vector describing a prototype document for that
cluster. Exemplar based methods restrict the cluster centers to the
observed documents, or documents that are \emph{exemplars} for the
clusters.  \\
\indent Given this objective--identify clusters of documents and
then documents that best define those clusters, we use two different
optimization appropriations, which take the similarity metric fixed:
kmediods and affinity propagation.
\subsubsection{Kmedoids}
Kmedoids is a greedy-optimization algorithm for exemplar based
clustering.  First, we fix a similarity metric: denote the
similarity between document $j$ and document $k$ as $s_{j,k}$. Then,
we initialize a set of $K$ documents as cluster centers
$\boldsymbol{\mu}_1, \hdots, \boldsymbol{\mu}_K$.  In the first step
of the algorithm, we compute the cluster that best fits each
document.  For document $i$ define the similarity with the
$k^{\text{th}}$ cluster center as $s_{i,k}$.  Then we can define the
first step as
\begin{eqnarray}
c_i & = & {\tt arg max }_k  s_{i,k} \nonumber
\end{eqnarray}
Then, based upon this step we update the cluster centers by choosing
the center that is at the ``center" of the cluster.  One way to
define the cluster center is to select the document with the highest
average similarity with the other documents in the cluster.
Formally, define the set of documents assigned to the
$k^{\text{th}}$ cluster as $d_k$.
 Then the average proximity with the other clusters for document $i$
 in cluster $k$ is $\bar{s}_{i,d_k}  =  \sum_{j \in d_k: j \neq i}
 \frac{s_{i,j} }{ |d_k| }$.  Then, we can define $\boldsymbol{\mu}_k
 = {\tt arg max}_{i \in d_k} \bar{s}_{i,d_k} $.  Table
 \ref{kmedoids} describes the procedure in full.
\begin{table}[hbt!]
\caption{Kmedoids Algorithm} \label{kmedoids} \framebox[7in]{
\begin{tabular}{l}
Initialize a set of cluster centers $\boldsymbol{\mu}_1, \hdots,
\boldsymbol{\mu}_K$  \\
Do until convergence \\
- \indent  for $i=1,\hdots,N$ $c_i  =  {\tt arg max }_k  s_{i,k}$ \\
- \indent for $k=1, \hdots, K$ set $d_k = \{ i : c_i = k \}$.  \\
- \indent for $i=1, \hdots, N$ and $k=1,\hdots, K$, compute
$\bar{s}_{i, d_k} = \sum_{j \in d_k : j \neq i }\frac{s_{i,j}}{|d_k|
}$ \\
- \indent for $k = 1,\hdots, K$ set $\boldsymbol{\mu}_k = {\tt arg
max}_{i \in d_k} \bar{s}_{i,d_k} $ \\
Return cluster labels $\boldsymbol{c}= (c_1, \hdots, c_N)$.
\end{tabular}}
\end{table}
\subsubsection{Affinity Propagation}
Affinity propagation is a different optimization procedure for
exemplar based clustering methods.  Rather than use a greedy
optimization algorithm, affinity propagation uses the \emph{max-sum
algorithm} on a \emph{factor-graph} to approximately consider all
possible solutions at the same time.  Affinity propagation does not
require an assumption about the number of clusters in the document,
rather the user sets only a prior on the number of clusters.  \\
\indent To outline the algorithm, we first fix a similarity metric
from above $s$.  To set the number of clusters in the data set, we
set the \emph{self-similarity} of each observation $s_{i,i}$ for all
$i=1,\hdots, N$.  As $s_{i,i}$ increases, there will be more
clusters in the data set.  Conversely, if $s_{i,i}$ is set lower,
there will be fewer clusters.  While we do not directly set the number of clusters from affinity-propagation, the self-similarity
exerts a strong influence on the number of clusters in the data.  \\
\indent Affinity propagation is a \emph{message-passing} algorithm,
where the messages will describe the \emph{responsibility} that an
observation serves as an exemplar for other observations, and the
\emph{availability} of an observation to serve as an exemplar.  To
begin the algorithm, define the availability matrix as
$\boldsymbol{A}$ as an $N \times N$ and set each element of
$\boldsymbol{A}$ equal to zero: typical element $a_{i,j} = 0$ to
start the algorithm.  Likewise the responsibility matrix,
$\boldsymbol{R}$ is $N \times N$.  With the fixed values of the
availability, define the responsibility $r_{i,k} $ as
\begin{eqnarray}
r_{i,k} & = & s_{i,k} - \max_{k^{'} \text{s.t.} k^{'} \neq k } \{
a_{i,k^{'}} + s_{i,k^{'}} \} \nonumber
\end{eqnarray}
Given this updated value of the $\boldsymbol{R}$ matrix, we update
the values of $\boldsymbol{A}$ matrix.  For document $i$, for all
observations $j \neq i$, we define the updates as
\begin{eqnarray}
a_{i,j} & = & \min\left(0, r_{j,j} + \sum_{i^{'} \text{s.t.} i^{'}
\notin \{i, k\}} \max \left(0, r_{i^{'}, k} \right) \right)
\nonumber
\end{eqnarray}
and define the self-availabilities as
\begin{eqnarray}
a_{i,i} & = & \sum_{i^{'} \text{s.t.} i^{'} \neq i } \max\{0,
r_{i^{'},i} \}  \nonumber
\end{eqnarray}
Dueck and Frey (2007) suggest \emph{dampening} the messages. Suppose
that index the interactions by $t$.  Then, suppose that we place the
updated messages into the matrix $\boldsymbol{R}^{\text{update}}$
and $\boldsymbol{A}^{\text{update}}$ and the old messages (t-1) in
$\boldsymbol{R}^{\text{old}}$ and $\boldsymbol{A}^{\text{old}}$. Set
$\lambda$ as the dampening factor, then
\begin{eqnarray}
\boldsymbol{R}^{t} & = & \lambda \boldsymbol{R}^{\text{update }} +
(1- \lambda ) \boldsymbol{R}^{\text{old }} \nonumber \\
\boldsymbol{A}^{t} & = & \lambda \boldsymbol{A}^{\text{update }} +
(1- \lambda ) \boldsymbol{A}^{\text{old }} \nonumber
\end{eqnarray}

To finish the algorithm, define the $i^{\text{th}}$ row of
$\boldsymbol{A}$ as $\boldsymbol{a}_i$ and $\boldsymbol{R}$ as
$\boldsymbol{r}_i$.  Then, define $c_i = {\tt arg max}_k
(\boldsymbol{a}_i^{\text{new}} + \boldsymbol{r}_i^{\text{new}}) $.
If $c_i = i$, then document $i$ is defined as an exemplar, or is
assigned to the cluster around document $k$.  Table \ref{affinity}
summarizes the algorithm.
\begin{table}[hbt!]
\caption{Affinity Propagation} \label{affinity} \framebox[7in]{
\begin{tabular}{l}
Initialize $\boldsymbol{A}^{\text{old}}$ to $\boldsymbol{0}$,
$\boldsymbol{R}^{\text{old}}$, and $\lambda$.
\\
Set $s_{i,i}$ for $i = 1,\hdots, N$.  \\
Do until convergence \\
- \indent for $i,j = 1, \hdots, N$ set $r_{i,j}^{\text{update}} =
s_{i,j} - \max_{k^{'} \text{s.t.} k^{'} \neq k } \{ a_{i,k^{'}} +
s_{i,k^{'}}
\}$ \\
- \indent for $i,j = 1, \hdots, N$ set $a_{i,j}^{\text{update}} =
\min\{0, r_{j,j} + \sum_{i^{'} \text{s.t.} i^{'} \notin \{i, k \}
}\max(0, r_{i^{'}, k}
)  \}$ \\
- \indent for $i = 1, \hdots, N$ set $a_{i,i}^{\text{update}} =
\sum_{i^{'}
\text{s.t.} i^{'} \neq k} \max\{0, r_{i^{'}, k}  \} $.  \\
- \indent set $\boldsymbol{R}^{\text{new}} = \lambda
\boldsymbol{R}^{\text{update}} + (1- \lambda)
\boldsymbol{R}^{\text{old}} $ and $\boldsymbol{A}^{\text{new}} =
\lambda \boldsymbol{A}^{\text{update}} + (1- \lambda)
\boldsymbol{A}^{\text{old}} $ \\
- \indent for $i = 1, \hdots, N$ set $c_i = {\tt arg max}_{k}
\left(\boldsymbol{a}_i^{\text{new}} + \boldsymbol{r}_i^{\text{new}} \right)$ \\
Return cluster labels $\boldsymbol{c} = (c_1, \hdots, c_N)$.
\end{tabular}}
\end{table}

\subsection{Kmeans}
Kmeans is probably the most well-known of clustering methods and is
the default clustering method for a variety of clustering tasks. The
algorithm proceeds in two basic steps, which are similar to exemplar
based methods.  The critical difference between kmeans and exemplar
based methods is that the cluster centers are now free to take on
any value, rather than just the documents.
\subsubsection{Classic Kmeans}
To state the algorithm, we fix a distance metric $\mathcal{D}_m$ and
initialize a set of cluster centers $\boldsymbol{\mu}_{1},\hdots,
\boldsymbol{\mu}_K$.  In the first step, we assign each document $i$
to a cluster center,
\begin{eqnarray}
c_i & = & {\tt arg min}_k \mathcal{D}(\boldsymbol{y}_i,
\boldsymbol{\mu}_k ) \nonumber
\end{eqnarray}

Conditional on the updated values of $\boldsymbol{c}$, we obtain
updated cluster centers.  Call $d_k = \{i : c_i = k \}$.  Then, the
updated cluster centers $\boldsymbol{\mu}_k$ is
\begin{eqnarray}
\boldsymbol{\mu}_k & = & \min_{\boldsymbol{\mu}_k \in \Re^{w} }
\sum_{j \in d_k } \mathcal{D}_m (\boldsymbol{y}_j,
\boldsymbol{\mu}_k ) \nonumber
\end{eqnarray}
with the specific value of $\boldsymbol{\mu}_k$ depending upon the
distance metric assumed.  The specific form of this update equation
will vary according to the distance metric, but the updates for
Euclidean distance are particularly straightforward,
\begin{eqnarray}
\boldsymbol{\mu}^{\text{new}}_k & = & \sum_{j \in d_k}
\frac{\boldsymbol{y}_j}{|d_k|} \nonumber
\end{eqnarray}
or the cluster centers are just the average of the data assigned to
the cluster.  Table \ref{kmeans} summarizes the algorithm

\begin{table}[hbt!]
\caption{Kmeans algorithm} \label{kmeans} \framebox[7in]{
\begin{tabular}{l}
Initialize $\boldsymbol{\mu}_1,\hdots, \boldsymbol{\mu}_K$ and
select $\mathcal{D}_m$ .  \\
Do until convergence \\
- \indent for $i =1, \hdots, N$ set $c_i = {\tt arg min}_k
\mathcal{D}_m (\boldsymbol{y}_i, \boldsymbol{\mu}_k)$.  \\
- \indent for $k =1, \hdots, K$ set $\boldsymbol{\mu}_k = {\tt arg
min}_{\boldsymbol{\mu}_k \in \Re^{w}} \sum_{j \in d_k} \mathcal{D}_m
(\boldsymbol{y}_j, \boldsymbol{\mu}_k) $ \\
Return $c_1, \hdots, c_N$ and $\boldsymbol{\mu}_1, \hdots,
\boldsymbol{\mu}_K$.
\end{tabular}}
\end{table}
\subsubsection{Fuzzy (soft) Kmeans} \textbf{You need to add this
method to the package (fuzzy you have
is a little different)} \\

A straightforward extension of the kmeans algorithm is the fuzzy
kmeans algorithm. Rather than searching for \emph{hard} assignments
to clusters, we allow the documents to be assigned partially to
other clusters.\footnote{The fuzzy comes from the idea of
\emph{fuzzy-set} membership, a generalization of set theory. }  \\
\indent Like kmeans, fuzzy kmeans is an iterative algorithm that
proceeds in two broad steps.  We fix a distance metric
$\mathcal{D}_m$ and a desired number of clusters $K$ and initialize
a set of cluster centers $\boldsymbol{\mu}_1, \hdots,
\boldsymbol{\mu}_K$.  In the first step we assign documents to
topics.  Now $\boldsymbol{c}_i$ is a $K \times 1$ vector, with
typical element $c_{i,j}$ given by,
\begin{eqnarray}
c_{i,j} & = & \frac{\left( \mathcal{D}^2_m (\boldsymbol{y}_i,
\boldsymbol{\mu}_j)\right)^{-\frac{1}{q-1}}}{\sum_{k=1}^{K}
\left(\mathcal{D}^2_m (\boldsymbol{y}_i, \boldsymbol{\mu}_k)
\right)^{-\frac{1}{q-1}} } \nonumber
\end{eqnarray}
where $q$ is a parameter that controls the \emph{fuzziness} of the
clusters.\footnote{The asymptotic behavior of $q$ is a little
complicated.  As $q \rightarrow - \infty$ or $q \rightarrow \infty$
then the documents are evenly assigned to each document.  As $q
\rightarrow 1$ from the left, the ratio behaves as the original
kmeans.  As $q \rightarrow 1$ from the right, the ratio behaves like
an inverted kmeans (assigning all weight to smallest document).}
Then,conditional on this step, we recompute the cluster centers by
calculating,
\begin{eqnarray}
\boldsymbol{\mu}_k & = & \frac{\sum_{i=1}{N} c_{i,k}^q
\boldsymbol{y}_i }{\sum_{i=1}^{N} c^{q}_{i,k} } \nonumber
\end{eqnarray}
Table \ref{fuzzyk} describes the procedure in full.
\begin{table}[hbt!]
\caption{Fuzzy Kmeans algorithm}\label{fuzzyk} \framebox[7in]{
\begin{tabular}{l}
Fix a distance metric $\mathcal{D}_m$, desired number of clusters
$K$ and fuzziness level $q$. \\
Do until convergence \\
 - For each $i=1, \hdots, N$ and $k=1,\hdots,
K$, compute $ c_{i,j} = \frac{\left( \mathcal{D}^2_m
(\boldsymbol{y}_i,
\boldsymbol{\mu}_j)\right)^{-\frac{1}{q-1}}}{\sum_{k=1}^{K}
\left(\mathcal{D}^2_m (\boldsymbol{y}_i, \boldsymbol{\mu}_k)
\right)^{-\frac{1}{q-1}} } $ \\
- For each $k=1,\hdots, K$ compute $ \boldsymbol{\mu}_k  =
\frac{\sum_{i=1}{N} c_{i,k}^q \boldsymbol{y}_i }{\sum_{i=1}^{N}
c_{i,k}^{q} } $ \\
Return $N \times K$ matrix of cluster labels $\boldsymbol{c}$.
\end{tabular}}
\end{table}

\subsubsection{Trimmed Kmeans}
{\tt This is a little hard to piece together from the paper.  The
basic idea is that this can robustify kmeans by trimming some of the
points that may be outliers}

\subsection{Maximum Entropy Clustering (MEC)}
Maximum Entropy Clustering (MEC) is a clustering method that
maximizes the entropy in cluster assignments, for an average
variance in the data set.  \\
\indent Maximum entropy clustering is defined only for the Euclidean
distance metric and we must assume the number of clusters in the
data set $K$.  To define the algorithm, we first fix a set of
cluster centers $\boldsymbol{\mu}_1, \hdots, \boldsymbol{\mu}_K$.
 Define the \emph{energy} for a cluster $\boldsymbol{\mu}_k$ and
document $\boldsymbol{y}_i$ as
\begin{eqnarray}
\text{E}_j(\boldsymbol{y}_i) & = &
\mathcal{D}^2_{\text{Euclidean}}(\boldsymbol{y}_i,
\boldsymbol{\mu}_k) \nonumber
\end{eqnarray}
In the first step of the algorithm, we determine the proportion of
the document that is assigned to each cluster, or compute a $K
\times 1$ vector $\boldsymbol{c}_i$, which has typical element
$c_{i,j}$,
\begin{eqnarray}
c_{i,j} & = & \frac{\exp\{- \beta \text{E}_j(\boldsymbol{y}_i)
\}}{\sum_{m=1}^{K} \exp\{ - \beta \text{E}_m(\boldsymbol{y})_i\}}
\end{eqnarray}
where $\beta$ is a parameter which controls the \emph{fuzziness} of
the updates.  As $\beta \rightarrow \infty$, then MEC approaches the
kmeans algorithm and $\beta \rightarrow 0$, every observation is
assigned equally to each document. $\beta$ can either be set a priori, or increased slowly during
estimation in an annealing procedure.   \\
\indent In the second step of the algorithm, we update the cluster
centers, based upon the updated values of $\boldsymbol{c}_i$.
Solving a constrained optimization problem shows that typical
cluster center $\boldsymbol{\mu}_k$ has update steps
\begin{eqnarray}
\boldsymbol{\mu}_k & = & \frac{\sum_{i=1}^{N} \boldsymbol{y}_i
c_{i,k} }{ \sum_{i=1}^{N} c_{i,k} } \nonumber
\end{eqnarray}
or the weighted average of the documents assigned to cluster $k$.
Table \ref{mec} summarizes the algorithm.

\begin{table}[hbt!]
\caption{Maximum Entropy Clustering Algorithm} \framebox[7in]{
\begin{tabular}{l}
Set number of clusters $K$, initialize $\beta$ and cluster centers
$\boldsymbol{\mu}_1, \hdots, \boldsymbol{\mu}_K$.  \\
Do until convergence \\
- for $i=1, \hdots, N$, $k=1,\hdots, K$ calculate
$\text{E}_j(\boldsymbol{y}_i)  =
\mathcal{D}^2_{\text{Euclidean}}(\boldsymbol{y}_i,
\boldsymbol{\mu}_k) $ \\
 - for $i=1, \hdots, N$, $k=1,\hdots, K$, calculate $c_{i,k} =
\frac{\exp\{- \beta \text{E}_j(\boldsymbol{y}_i) \}}{\sum_{m=1}^{K}
\exp\{ - \beta \text{E}_m(\boldsymbol{y})_i\}}$ \\
- for $k =1, \hdots, K$ calculate $\boldsymbol{\mu}_k  =
\frac{\sum_{i=1}^{N} \boldsymbol{y}_i c_{i,k} }{ \sum_{i=1}^{N}
c_{i,k} }$ \\
Return $N \times K$ matrix of cluster memberships $\boldsymbol{c}$
\end{tabular}}
\end{table}



\subsection{Distance Cuts}
\textbf{This function first forms a graph based upon a threshold
$\beta$.  Then this graph is cut to identify clusters or
\emph{subgraphs}.  No citation or code given, so you'll need to
unpack the .C code in the CBA library. }



\subsection{Hierarchical Methods}


\subsubsection{Agglomerative Hierarchical Clustering (AHC)}
\subsubsection{(AHC): Rock}
{\tt pg 207} of the Data Clustering, Theory, and Applications book.
And located here \\ {\tt
http://www.cis.upenn.edu/~sudipto/mypapers/categorical.pdf}.  You
need to figure out what a \emph{heap} is.
\subsubsection{(AHC): Mixture of Normals}
\subsubsection{Divisive Hierarchical Clustering (DHC)}
\subsubsection{(DHC): DISMEA}
This is a divisive clustering method, that successively looks for
divisions to make in the clusters until we reach the desired number
of clusters in the data set.  \\
\indent To start the algorithm, we fix a distance metric
$\mathcal{D}_m$ and a desired number of clusters $K$.  Recall that
$d_1 \equiv \{i: c_i = 1\}$.  Then, the first step of the algorithm
is to look for sets $d_1$ and $d_2$ such that
\begin{eqnarray}
\text{F}(d_1, d_2,\boldsymbol{\mu}_1, \boldsymbol{\mu}_2,
\boldsymbol{Y}) & = & \min_{d_1, d_2} \sum_{i=1}^{2} \sum_{j \in
d_i} \mathcal{D}_m (\boldsymbol{y}_j, \boldsymbol{\mu}_i) \nonumber
\end{eqnarray}
where $\boldsymbol{\mu}_i$ is the cluster center for the
$i^{\text{th}}$ cluster.  To search for the division, we use the
kmeans algorithm looking for two clusters.  In subsequent steps, we
repeat the algorithm, but on the cluster with the \emph{greatest
dispersion} around the cluster center.  Calculate the dispersion for
cluster $k$ as
\begin{eqnarray}
\text{Dispersion}_k & = & \sum_{j \in d_k}
\mathcal{D}(\boldsymbol{y}_j, \boldsymbol{\mu}_k). \nonumber
\end{eqnarray}
We then proceed to split the cluster $j = {\tt arg max}_k
\text{Dispersion}_k$.  We repeat the procedure until there are $K$
clusters remaining.  Table \ref{dismea} summarizes the algorithm.
\begin{table}[hbt!]
\caption{DISMEA algorithm} \label{dismea} \framebox[7in]{
\begin{tabular}{l}
Fix a distance metric $\mathcal{D}_m$ and number of clusters $K$. \\
Do until $K$ clusters are obtained \\
- For each $k=1, \hdots, m$ calculate $\text{Dispersion}_k   =
\sum_{j \in d_k} \mathcal{D}(\boldsymbol{y}_j, \boldsymbol{\mu}_k)$.
\\
- Set $j = {\tt arg max}_k \text{Dispersion}_k$. \\
- Apply kmeans to $d_j$ with $k=2$.  Call $d_1= d_j$ and $d_2 =
d_{m+1}$
\end{tabular}
}
\end{table}

\subsection{Fuzzy Clustering}
\textbf{This is the FANNY function in the \emph{cluster package.}  }

\subsection{Hard Clustering}

\subsection{Markov Clustering Algorithm}
Check this algorithm out.  Supposedly it is conceptually similar to
Affinity propagation, but you'll have to check it out. \\
\textbf{This algorithm is passed on repeated application of matrix
product and elementwise exponentiating.  The result (if properly
applied) is a set of star graphs where the nodes at the center of
the graph are functionally similar to the exemplars in affinity
propagation.  }

\subsection{Clustering Convex Functions}

\subsection{Quality Threshold Clustering (QTClust)}
Quality Threshold Clustering (QTCLust) is a clustering method that
relies upon identifying groups of observations that lie within a
threshold $d$, rather than explicitly identifying cluster centers.
\\
\indent To outline the algorithm, we first fix a distance metric
$\mathcal{D}_m$ and a desired number of clusters $K$.  Then, for
each document $i$ we determine the number of documents that are
within $d$ the \emph{threshold} and group all such documents into
the set $A_i$. Then we choose


\subsection{Neural Gas}

\subsection{Self-Organizing Map}
\subsection{Self-Organizing Tree}

\subsection{Spectral Clustering}
Spectral clustering is an approach to clustering that involves
working with the \emph{spectrum}--the eigenvalues and
eigenvectors--of similarity matrices.  By working on the spectrum,
spectral methods may be effective at changing the representation of
documents--allowing the identification of clusters that would be
otherwise impossible to identify using standard methods.\\
\indent A variety of spectral methods have been proposed, but all
have some similar basic elements.  First, they are based upon a
similarity matrix $\boldsymbol{S}$, which is set \emph{a priori}
from the list above.  Second, a \emph{graph laplacian},
$\boldsymbol{L}$ is constructed, which is a reweighted similarity
matrix (described below).  Using these objects, we can define three
different spectral clustering algorithms.\\
\indent To define the algorithms, we will need to order the
eigenvectors and eigenvalues.  Throughout, we will organize the
vectors by their size of their associated eigenvalues.  The
eigenvector associated with the largest eigenvalues will be the
\emph{first} eigenvector, the eigenvector with the second largest
eigenvalue will the \emph{second} eigenvector and so on.\\
\indent \textbf{You can expand this, or at least make more clear}.
You can justify the use of spectral clustering in three ways.  (1)
You can use a graph-cut justification, where you try to cut the
edges to create clusters that have the lowest similarity. (2) You
can use a random-walk justification, where we want to segment the
sections where the random-walk is likely to spend the most time. (3)
Use a perturbation argument.  You can show that in the ideal case,
the spectral clustering algorithm is likely to identify the correct
solution.  Then, as you perturb the ideal solution, the spectral
method is likely to find the correct solution.  Note, that in all
three cases, using the eigenvalues \emph{relaxes} the optimization
problem, making it simpler than the true problem.  In some
degenerate graphs, this can lead to some problems.
\subsubsection{Unnormalized Spectral Clustering}
\textbf{add this method to the package} To construct this
clustering, we first need to define the \emph{degree} of each
observation.  For each observation $i=1,\hdots, N$ define $d_i =
\sum_{j=1}^{N} s_{i,j}$.  Define $\boldsymbol{D}$ as the $N \times
N$ diagonal matrix with the degrees down the diagonal,
\begin{equation}
\boldsymbol{D} = \begin{pmatrix}
d_1 & 0 & \hdots & 0 \\
0   & d_2 & \hdots & 0 \\
\vdots   &  \vdots  & \ddots & \vdots \\
0       &     0     &   \hdots & d_N \\
\end{pmatrix} \nonumber
\end{equation}
Then, define $\boldsymbol{L} = \boldsymbol{D} - \boldsymbol{S}$.  If
we want to compute $K$ clusters, then we obtain the first $K$
eigenvectors associated with $\boldsymbol{L}$ and place them as the
columns in the $N \times K$ matrix $\boldsymbol{U}$.  Call the rows
of $\boldsymbol{U}$, $u_1, \hdots, u_N$ and use the kmeans
algorithm, with the euclidean metric assumed, to cluster the rows of
$\boldsymbol{U}$.  We then say that $c_i = k $ if and only if $u_i$
is assigned to the $k^{\text{th}}$ cluster.  Table \ref{unnormspec}
summarizes the algorithm.
\begin{table}[hbt!]
\caption{Unnormalized Spectral Clustering} \label{unnormspec}
\framebox[7in]{
\begin{tabular}{l}
Fix a similarity matrix $\boldsymbol{S}$ and number of clusters $K$. \\
- Calculate $\boldsymbol{D}$ and $\boldsymbol{L} = \boldsymbol{D}-
\boldsymbol{S}$.  \\
- Compute first $K$ eigenvectors of $\boldsymbol{L}$ and place as
columns in $N \times K$ matrix $\boldsymbol{U}$.  \\
- Call rows of $\boldsymbol{U}$ $u_1, \hdots, u_N$.  \\
- Cluster rows of $\boldsymbol{U}$ using kmeans and euclidean
distance.  \\
- Assign $c_i = k $ if and only if $u_i$ is assigned to
$k^{\text{th}}$ cluster.  \\
Return $\boldsymbol{c}$.
\end{tabular}}
\end{table}
\subsubsection{Meila and Shi (2002) Spectral Clustering }
This algorithm has the same basic steps as unnormalized spectral
clustering, but computes the spectrum of a different laplacian
matrix $\boldsymbol{L}$.  Once again, we fix a similarity matrix
$\boldsymbol{S}$ and compute the degree matrix $\boldsymbol{D}$.
 Define
\begin{eqnarray}
\boldsymbol{D}^{-1} & = & \begin{pmatrix}\frac{1}{d_1} & 0 & \hdots
& 0 \\
0 & \frac{1}{d_2}& \hdots & 0 \\
\vdots & \vdots & \ddots & \vdots \\
0 & 0 & \hdots & \frac{1}{d_N}
\end{pmatrix} \nonumber
\end{eqnarray}

We can define the new graph Laplacian, $\boldsymbol{L}^{'}$, as
\begin{eqnarray}
\boldsymbol{L}^{'} & = & \boldsymbol{D}^{-1} (\boldsymbol{D} -
\boldsymbol{S}) \nonumber \\
& = & \boldsymbol{I} - \boldsymbol{D}^{-1}\boldsymbol{S}, \nonumber
\end{eqnarray}
where $\boldsymbol{I}$ is the $N \times N$ identity matrix.  We then
compute the first $K$ eigenvectors of $\boldsymbol{L}^{'}$, place
those vectors in the $N \times K$ matrix $\boldsymbol{U}$, and then
cluster the rows $\boldsymbol{u}_1, \hdots, \boldsymbol{u}_N$ using
a kmeans algorithm with euclidean distance.  We then cluster the
documents based upon the clustering of the rows of the
$\boldsymbol{U}$ matrix.  Table \ref{MeilaShi} describes the
algorithm.

\begin{table}[hbt!] \caption{Meila and Shi (2001) Spectral
Clustering} \label{MeilaShi} \framebox[7in]{
\begin{tabular}{l}
Fix a similarity matrix $\boldsymbol{S}$ and number of clusters $K$.
\\
- Calculate $\boldsymbol{D}$. \\
- Compute $\boldsymbol{L}^{'} = \boldsymbol{I} - \boldsymbol{D}^{-1}
\boldsymbol{S}$ \\
- Calculate the first $K$ eigenvectors of $\boldsymbol{L}^{'}$ and
place as columns in $N \times K$ matrix $\boldsymbol{U}$.\\
- Cluster rows of $\boldsymbol{U}$ using kmeans with euclidean
distance.  \\
- Assign $c_i = k $ if and only if $u_i$ is assigned to
$k^{\text{th}}$ cluster.  \\
Return $\boldsymbol{c}$.
\end{tabular}}
\end{table}


\subsubsection{Ng, Jordan, Weiss (2003) Spectral Clustering}
Once again, this algorithm follows the same basic structure as the
previous two--with a slightly different graph Laplacian used, and an
additional normalization step. \\
\indent Once again, we fix a similarity matrix $\boldsymbol{S}$ and
a desired number of clusters $K$.  Define $\boldsymbol{D}^{-1/2}$
as,
\begin{eqnarray}
\boldsymbol{D}^{-1/2} & = & \begin{pmatrix}\frac{1}{\sqrt{d_1}} & 0
& \hdots
& 0 \\
0 & \frac{1}{\sqrt{d_2}}& \hdots & 0 \\
\vdots & \vdots & \ddots & \vdots \\
0 & 0 & \hdots & \frac{1}{\sqrt{d_N}}
\end{pmatrix} \nonumber
\end{eqnarray}

We then compute a different version of the Graph Laplacian
$\boldsymbol{L}^{''}$ as,
\begin{eqnarray}
\boldsymbol{L}^{''} & = & \boldsymbol{D}^{-1/2} \boldsymbol{S}
\boldsymbol{D}^{-1/2} \nonumber \\
& = & \boldsymbol{I} - \boldsymbol{D}^{-1/2} \boldsymbol{S}
\boldsymbol{D}^{-1/2} \nonumber
\end{eqnarray}

Now, we compute the first $K$ eigenvectors of $\boldsymbol{L}^{''}$
and place the vectors in the $N \times K$ matrix $\boldsymbol{U}$.
Next, we form the normalized matrix $\boldsymbol{U}^{\text{norm}}$
by normalizing the rows $\boldsymbol{U}$.  That is row
$\boldsymbol{u}^{\text{norm}}_{i}$ is equal to,
\begin{eqnarray}
\boldsymbol{u}^{\text{norm}}_{i} & = & \frac{\boldsymbol{u}_i
}{\sqrt{\boldsymbol{u}_i^{'} \boldsymbol{u}^{'} }} \nonumber
\end{eqnarray}
Now, we cluster the rows of $\boldsymbol{U}^{\text{norm}}$ using a
kmeans algorithm with Euclidean distance metric and assign the
document clusters based upon the clusters found in the rows of
$\boldsymbol{U}^{\text{norm}}$.  Table \ref{NgJordanWeiss}
summarizes the algorithm.
\begin{table}[hbt!]
\caption{Ng, Jordan, and Weiss (2003) Spectral Clustering}
\label{NgJordanWeiss} \framebox[7in]{
\begin{tabular}{l}
Fix a similarity matrix $\boldsymbol{S}$ and desired number of
clusters $K$.  \\
- compute $\boldsymbol{L}^{''} = \boldsymbol{I} -
\boldsymbol{D}^{-1/2} \boldsymbol{S} \boldsymbol{D}^{-1/2}$ \\
- compute the first $K$ eigenvectors of $\boldsymbol{L}^{''}$ and
place them as the columns in $\boldsymbol{U}$.  \\
- compute $\boldsymbol{U}^{\text{norm}}$ by normalizing each row of
$\boldsymbol{U}$, $\boldsymbol{u}^{\text{norm}}_i =
\frac{\boldsymbol{u}_i }{\sqrt{\boldsymbol{u}_i^{'}
\boldsymbol{u}^{'} }}$.  \\
- Cluster rows of $\boldsymbol{U}^{\text{norm}}$ using kmeans with
Euclidean
distance.  \\
- Assign $c_i = k $ if and only if $u_i^{\text{norm}}$ is assigned
to $k^{\text{th}}$ cluster.  \\
Return $\boldsymbol{c}$.
\end{tabular}}
\end{table}

\subsubsection{Shi and Malik (2000) Spectral Clustering}
\textbf{expand this when you work on it again}Rather than just work
directly on a Laplacian, this algorithm segments the data based upon
the size of the second eigenvector of the graph Laplacian,
iteratively dividing the data into $K$ clusters.  To begin, we fix a
similarity matrix $\boldsymbol{S}$ and fix the number of clusters
$K$.  Then, we compute $\boldsymbol{L}^{'}$ and compute the
eigenvalues and eigenvectors. We take the second eigenvector, and
rank the data based upon the value of the second eigenvector. We
then go through and find the partition of the observations $i=
1,\hdots, N-1$ that minimizes the $NCUT$ between the two clusters,
which for set $A$ and $B$ is defined as,
\begin{eqnarray}
NCUT(A, B) & = & Cut(A, B) \left(\frac{1}{\text{vol} A } +
\frac{1}{\text{vol} B} \right) \nonumber
\end{eqnarray}
where $Cut(A,B) = sum_{i \in A}\sum_{j \in B} s_{i,j}$, vol$A =
\sum_{i \in A} d_i$ and vol $B = \sum_{j \in B} d_j$.  We then
repeat for the cluster with the largest second eigenvalue until we
have $K$ clusters.

\subsection{Statistical Models}

\subsubsection{Dirichlet Process prior, Multinomial Components}
\subsubsection{Mixture of Multinomial Distributions, EM}
\subsubsection{Mixture of Multinomial Distributions, Variational Estimation}
\subsubsection{Mixture of von-Mises Fisher, EM}
\subsubsection{Mixture of von-Mises Fisher, Variational Estimation}
\subsubsection{Mixture of Normals}
\subsubsection{Mixture of Normals, Variational Estimation}

\subsection{Coclustering}


\subsubsection{Coclustering with Mutual Information}
\subsubsection{Coclustering with Spectral Methods (SVD)}


\end{document}
